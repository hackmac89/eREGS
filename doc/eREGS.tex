% !TEX encoding = UTF-8 Unicode
\documentclass{article}
\usepackage[utf8]{inputenc}   % Umlaute automatisch übernehmen
\usepackage{listings}   % für Codelistings
\usepackage{color}   % für das Syntaxhighlighting
\usepackage{graphicx}   % um Bilder einzufügen
\usepackage[colorlinks=true]{hyperref}   % Um Hyperlinks zu unterstützen (farbige Links als Option)
\usepackage{url}   % um url in footnotes formatiert anzuzeigen

\definecolor{mygreen}{rgb}{0,0.6,0}
\definecolor{mygray}{rgb}{0.5,0.5,0.5}
\definecolor{mymauve}{rgb}{0.58,0,0.82}
 
\lstset{ %
  backgroundcolor=\color{white},   % Hintergrundfarbe
  basicstyle=\ttfamily\scriptsize,   % Schriftgröße für den Code
  breakatwhitespace=false,   % autom. Zeileneinbruch bei Leerzeichen deaktivieren 
  breaklines=true,   % autom. Zeilenumbruch
  captionpos=b,   % caption-position = bottom
  commentstyle=\color{mygreen},   % Kommentarstil
  escapeinside={\%*}{*)},   % Falls LaTex innerhalb des Codes verwendet werden soll
  extendedchars=true,   % um mittels lstlisting auch utf-8 Dateien und Strings lesen zu können
  literate={Ä}{{\"A}}1 {Ö}{{\"O}}1 {Ü}{{\"U}}1 {ä}{{\"a}}1 {ö}{{\"o}}1 {ü}{{\"u}}1 {ß}{{\ss}}1, % um mittels lstlisting auch utf-8 Dateien und Strings lesen zu können
  frame=single,   % Rahmen um den Code packen
  inputencoding=utf8,
  keepspaces=true,   % Leerzeichen im Code beibehalten, nützlich für vorh. Einrückungen
  keywordstyle=\color{blue},   % Stil der Schlüsselwörter
  language=C,   % Programmiersprache festlegen
  numbers=left,   % Zeilennummer links ausgeben (none, left, right)
  numbersep=5pt,   % Abstand der Zeilennummern zum Code
  numberstyle=\tiny\color{mygray},   % Stil der Zeilennummern
  rulecolor=\color{black},
  showspaces=false,
  showstringspaces=false,          
  showtabs=false,         
  stepnumber=2,  
  stringstyle=\color{mymauve},   % string literal stil
  tabsize=2,   % Standardgröße der Tabs =  2 Leerzeichen
  title=\lstname   % Dateinamen der inkludierten Dateien mit \lstinputlisting zeigen 
}

\begin{document}

\title{Stackmaschine eREGS}
\author{Jeff Wagner}
\maketitle

\begin{abstract}
Erstellen einer Sprache für einen Stackautomaten\footnote{\url{https://de.wikipedia.org/wiki/Kellerautomat}} und eines in C geschriebenen Interpreters
\end{abstract}

\section{Vorwort}
\label{sec:vorwort}
Es wird eine Sprache, \textit{"eREGS"} genannt, entworfen.\\ 
Des Weiteren wird mittels der Programmiersprache C und den Scanner- und Parsergeneratoren \textit{"flex"}\footnote{\url{https://de.wikipedia.org/wiki/Lex_(Informatik)}} respektive \textit{"yacc/bison"}\footnote{\url{https://de.wikipedia.org/wiki/Yacc}} ein Programm entworfen, 
welches eine Eingabe-Code-Datei interpretiert und ausführt.\\
Es wird \textbf{kein} Ausgabemodul geschrieben, welches z.B. den Code für eine spezifische Prozessorarchitektur übersetzt, auf derer das Kompilat einzig auszuführen wäre.
\\\\
Bevor es in Abschnitt~\ref{sec:eregsLang} ab Seite \pageref{sec:eregsLang} mit einer detaillierten Erläuterung der Sprache weitergeht, wird im Folgenden noch das 
Code-Layout~\ref{sec:codelayout} und die Inbetriebnahme 
des Projektes~\ref{sec:programmablauf} vorgestellt.
\newpage

\section{Einleitung}
\label{sec:einleitung}
Das Programm ist nach folgender Konvention nur von der Kommandozeile auszuführen:
\paragraph{Aufruf (Linux/MAC): } ./eregs Code-Eingabedatei
\paragraph{Aufruf (Windows): } eregs.exe Code-Eingabedatei 

\hfill\break
\subsection{Code-Layout}
\label{sec:codelayout}
Das Projekt hat folgenden Aufbau:\\
\\
\paragraph{bin/}\hspace{4em}Ordner mit dem von der MAKEFILE erstelltem Programm
\paragraph{doc/}\hspace{4em}LaTeX Quellcode und PDF mit dieser Dokumentation
\paragraph{examples/}\hspace{1em}Beispielprogramme in der "eRegs" Sprache
\paragraph{main.l}\hspace{3em}Programm-Einstiegspunkt und Scanner-Definition
\paragraph{makefile}\hspace{2em}Automatische Programmerzeugung mit "make" tool
\paragraph{parser.y}\hspace{2em}Parser und Interpreter-Implementierung
\\
\subsection{Kompilierung \& Programmablauf}
\label{sec:programmablauf}
Zum Kompilieren genügt es, mit der Kommandozeile in das Verzeichnis zu gehen, welches die "makefile" beinhaltet (also das Stamm-Projektverzeichnis) und \textit{\textbf{make}} einzugeben. Anhand dessen wird bei erfolgreicher Kompilierung das Programm im \textit{\textbf{bin}}-Ordner erzeugt.\\
Um ein in "eRegs" geschriebenes Programm zu starten, muss der Interpreter mit der Code-Datei als Argument von der Kommandozeile gestartet werden.\\
Siehe dazu Abschnitt~\ref{sec:einleitung}.
\\\\

\section{Die Sprache eREGS}
\label{sec:eregsLang}
Die Sprache "eRegs" unterstützt arithmetische Operationen sowie Verzweigungen und Schleifen.\\
In der ALPHA werden momentan bis zu 32 Register (quasi Variablen) und 999 Sprungmarken (für Verzweigungen und Schleifen) unterstützt.\\
Der Stack kann ebenso simultan maximal 999 Werte zwischenspeichern.\\
Des Weiteren werden innerhalb des Quelltextes Kommentare\\ im C Stil (\textit{\textbf{//}} für Einzeilige Kommentare und \textit{\textbf{/* ... */}} für Mehrzeilige Kommentare) unterstützt.\\\\
Der genaue Befehlssatz umfasst:\\\\

\begin{tabular}{|c|c|}\hline
   \textbf{push} & Legt Element auf die Spitze des Stacks \\ \hline
   \textbf{pop} & Holt oberstes Element vom Stack \\ \hline
   \textbf{add} & Holt die 2 obersten Elemente vom Stack, addiert diese und legt das Ergebnis auf den Stack \\ \hline
   \textbf{sub} & Wie "add", nur für Subtraktion \\ \hline
   \textbf{mul} & Wie "add", nur für Multiplikation \\ \hline
   \textbf{div} & Wie "add", nur für Division (prüft Division mit Null) \\ \hline
   \textbf{cmpl} & Prüft die 2 obersten Elemente (setzt 1, falls Arg1 $<$  Arg2), 0 ansonten \\ \hline
   \textbf{cmpg} & Prüft die 2 obersten Elemente (setzt 1, falls Arg1 $>$  Arg2), 0 ansonten  \\ \hline         
   \textbf{cmple} & Prüft die 2 obersten Elemente (setzt 1, falls Arg1 $<=$ Arg2), 0 ansonten  \\ \hline        
   \textbf{cmpge} & Prüft die 2 obersten Elemente (setzt 1, falls Arg1 $>=$  Arg2), 0 ansonten  \\ \hline 
   \textbf{cmpeq} & Prüft die 2 obersten Elemente (setzt 1, falls Arg1 $==$  Arg2), 0 ansonten  \\ \hline      
   \textbf{cmpne} & Prüft die 2 obersten Elemente (setzt 1, falls Arg1 $!=$  Arg2), 0 ansonten  \\ \hline      
   \textbf{print} & Holt das oberste Element vom Stack und gibt es auf der Konsole aus \\ \hline       
   \textbf{jmp} & Unbedingter Sprung zu Sprungmarke \textbf{L}\textit{xxx}  \\ \hline         
   \textbf{jfalse} & Bedingter Sprung zu Sprungmarke \textbf{L}\textit{xxx} (Prüft Stack) \\ \hline
   \textbf{L} & Definition einer Sprungmarke, z.B. \textbf{L}\textit{1}:  \\ \hline           
   \textbf{r} & Definition eines Registers, z.B. \textbf{r}\textit{1} \\ \hline
   \textbf{shl} & Binärer \textit{Linksshift} um \textit{n} Stellen ($n * 2^n$) \\ \hline           
   \textbf{shr} & Binärer \textit{Rechtsshift} um \textit{n} Stellen ($n / 2^n$) \\ \hline         
   \textbf{rol} & Binäre \textit{Linksrotation} \\ \hline          
   \textbf{ror} & Binäre \textit{Rechtsrotation} \\ \hline
   \textbf{not} & Binäre \textit{Negation} \\ \hline
   \textbf{and} & Binäres \textit{UND} \\ \hline 
   \textbf{xor} & Binäres \textit{Exklusiv-ODER} \\ \hline		
   \textbf{mod} & Modulo Restoperator \\ \hline
\end{tabular}
\\\\
\\Genug der Theorie, in Abschnitt~\ref{sec:examples} geht es nun an ein paar Beispiele.\\
Eine Definition der Grammatik der Sprache WIRD NOCH NACHGEREICHT...\\

\section{Programmbeispiele}
\label{sec:examples}
\subsection{Do-While-Schleife}
% Beginn des Code-Blocks
\lstinputlisting{../examples/example1.r}

\subsection{Additionsbeispiel}
% Beginn des Code-Blocks
\lstinputlisting{../examples/example2.r}

\newpage
\subsection{Berechnung einer mathematischen Reihe}
% Beginn des Code-Blocks
\lstinputlisting{../examples/example3.r}
\newpage
\subsection{Berechnung der Fakultät $n!$}
% Beginn des Code-Blocks
\lstinputlisting{../examples/example5.r}
\newpage
\subsection{Shift Beispiele}
% Beginn des Code-Blocks
\lstinputlisting{../examples/example6.r}
\newpage
\subsection{Berechnung des Durchschnitts 2er Zahlen}
% Beginn des Code-Blocks
\lstinputlisting{../examples/example9.r}

Für weitere \textit{eventuell vorhandene} Beispiele siehe das \textit{\textbf{examples}}-Verzeichnis.

\end{document}